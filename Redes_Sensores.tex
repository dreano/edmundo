\documentclass[notes]{beamer}
\usepackage{graphicx}
\usepackage{url}

\mode<presentation>
{
  % A tip: pick a theme you like first, and THEN modify the color theme, and then add math content.
  % Warsaw is the theme selected by default in Beamer's installation sample files.

  %%%%%%%%%%%%%%%%%%%%%%%%%%%% THEME
%\usetheme{AnnArbor}
%\usetheme{Antibes}
%\usetheme{Bergen}
%\usetheme{Berkeley}
%\usetheme{Berlin}
%   \usetheme{Boadilla}
%\usetheme{boxes}
%\usetheme{CambridgeUS}
%   \usetheme{Copenhagen}
%   \usetheme{Darmstadt}
%\usetheme{default}
%\usetheme{Dresden}
   \usetheme{Frankfurt}
%\usetheme{Goettingen}
%\usetheme{Hannover}
%\usetheme{Ilmenau}
%\usetheme{JuanLesPins}
%\usetheme{Luebeck}
%\usetheme{Madrid}
%\usetheme{Malmoe}
%\usetheme{Marburg}
%\usetheme{Montpellier}
%\usetheme{PaloAlto}
%\usetheme{Pittsburgh}
%\usetheme{Rochester}
%\usetheme{Singapore}
%\usetheme{Szeged}
%\usetheme{Warsaw}

  %%%%%%%%%%%%%%%%%%%%%%%%%%%% COLOR THEME
  %\usecolortheme{albatross}
  %\usecolortheme{beetle}
  %\usecolortheme{crane}
  %\usecolortheme{default}
  %\usecolortheme{dolphin}
  %\usecolortheme{dove}
  %\usecolortheme{fly}
  %\usecolortheme{lily}
  \usecolortheme{orchid}
  %\usecolortheme{rose}
  %\usecolortheme{seagull}
  %\usecolortheme{seahorse}
  %\usecolortheme{sidebartab}
  %\usecolortheme{structure}
  %\usecolortheme{whale}

  %%%%%%%%%%%%%%%%%%%%%%%%%%%% OUTER THEME
  %\useoutertheme{default}
  %\useoutertheme{infolines}
  %\useoutertheme{miniframes}
  %\useoutertheme{shadow}
  %\useoutertheme{sidebar}
  %\useoutertheme{smoothbars}
  %\useoutertheme{smoothtree}
  %\useoutertheme{split}
  %\useoutertheme{tree}

  %%%%%%%%%%%%%%%%%%%%%%%%%%%% INNER THEME
  %\useinnertheme{circles}
  %\useinnertheme{default}
  %\useinnertheme{inmargin}
  %\useinnertheme{rectangles}
  %\useinnertheme{rounded}

  %%%%%%%%%%%%%%%%%%%%%%%%%%%%%%%%%%%

  \setbeamercovered{transparent} % or whatever (possibly just delete it)
  % To change behavior of \uncover from graying out to totally invisible, can change \setbeamercovered to invisible instead of transparent. apparently there are also 'dynamic' modes that make the amount of graying depend on how long it'll take until the thing is uncovered.

}


% Get rid of nav bar
\beamertemplatenavigationsymbolsempty

% Use short top
\usepackage[headheight=12pt,footheight=12pt]{beamerthemeboxes}
%\addheadboxtemplate{\color{black}}{
%\hskip0.3cm
%\color{white}
%\insertshortauthor \ \ \ \ 
%\insertframenumber \ \ \ \ \ \ \ 
%\insertsection \ \ \ \ \ \ \ \ \ \ \ \ \ \ \ \ \  \insertsubsection
%\hskip0.3cm}
%\addheadboxtemplate{\color{black}}{
%\color{white}
%\ \ \ \ 
%\insertsection
%}
%\addheadboxtemplate{\color{black}}{
%\color{white}
%\ \ \ \ 
%\insertsubsection
%}

% Insert frame number at bottom of the page.
\usefoottemplate{\hfil\tiny{\color{black!90}\insertframenumber}} 
\setbeamertemplate{footline}[frame number]
\usepackage[english, portuguese]{babel}
\usepackage[latin1, utf8]{inputenc}

\usepackage{times}
\usepackage[T1]{fontenc}

\title{Redes Sensores}
\subtitle{}

\author{Adriano Ricardo Ruggero}

\institute{Instituto de Computação - Unicamp}

\date{\today}

\subject{Talks}

\def\defn#1{{\color{red} #1}}

\begin{document}

\begin{frame}
\label{slide_1}
  \titlepage
\end{frame}

\begin{frame}
\label{slide_2}
  \frametitle{Agenda}
  \tableofcontents
\end{frame}

\section{Introdução}
\begin{frame}
\label{slide_3}
\frametitle{Introdução}

\begin{block}

 \center \textbf{Introdução}
 
\end{block}

\end{frame}

\begin{frame}
\label{slide_4}
\frametitle{Introdução}
\framesubtitle{Definições:} 

\begin{block}

Uma rede sem fio formada por um grande número de sensores pequenos e imóveis  que detectam e transmitem alguma característica física do ambiente. A informação contida nos sensores é agregada numa base central de dados \cite{Malladi_1}

\end{block}

\end{frame}

\begin{frame}
\label{slide_5}
\frametitle{Introdução}
\framesubtitle{Definições:} 

\begin{block}

Uma classe particular de sistemas distribuídos, onde as comunicações de baixo
nível não dependem da localização topológica da rede \cite{Huang_2}

\end{block}

\end{frame}

\begin{frame}
\label{slide_6}
\frametitle{Introdução}
\framesubtitle{Definições:} 
\begin{block}

Um conjunto de nós individuais (sensores) que operam sozinhos, mas que podem formar uma rede com o objetivo de juntar as informações individuais de cada sensor para monitorar algum fenômeno.

\end{block}

\end{frame}

\begin{frame}
\label{slide_7}
\frametitle{Introdução}
\framesubtitle{Objetivos}

\begin{block}

O objetivo principal é monitorar fenômenos em ambientes perigosos, de difícil acesso ou ambientes de interação direta com um indivíduo e transmiti-los em forma de dados. 
\end{block} \pause

\begin{block}

Cada nó sensor gera e armazena dados independentemente dos outros.

\end{block} \pause

\begin{block}

Apesar de menor e menos confiável do que equipamentos de rede tradicionais, juntos são capazes de monitorar fenômenos complexos. 

\end{block}

\end{frame}

\begin{frame}
\label{slide_8}
\frametitle{Introdução}
\framesubtitle{Vantagens}

\begin{block}

\begin{itemize}

\item O uso de tecnologias de rede já existentes confere um custo menor e melhor desempenho ao sistema \pause

\item Permite o monitoramento de locais perigosos e de difícil acesso \pause

\item A combinação de sensores com diferentes frequências confere maior precisão às medidas coletadas (desde que o posicionamento dos sensores seja preciso e haja sincronização entre eles) 

\end{itemize}

\end{block}

\end{frame}

\begin{frame}
\label{slide_9}
\frametitle{Introdução}
\framesubtitle{Usos}

\begin{block}

A área de Redes de Sensores sem Fio têm mostrado um grande potencial em fazer parte da vida das pessoas no futuro, levando a uma “simbiose” cada vez maior entre a máquina e o homem\cite{Redes_Sensores}. 

\end{block} \pause

\begin{block}

Observa-se isto pela gama de áreas em que redes de sensores podem ser utilizadas: \pause

\begin{itemize}

\item Medicina \pause
\item Tráfego urbano \pause
\item Indústrias \pause
\item Controle da poluição \pause
\item Estudo e prevenção de desastres naturais \pause
\item Automação doméstica \pause
\item Ambientes inteligentes 

\end{itemize}

\end{block}

\end{frame}

\section{Características}
\begin{frame}
\label{slide_10}
\frametitle{Características}

\begin{block}

 \center \textbf{Características}
 
\end{block}

\end{frame}

\begin{frame}
\label{slide_11}
\frametitle{Características}
\framesubtitle{Sensor}

\begin{block}

Dispositivo que monitora fisicamente um fenômeno ambiental e gera
relatórios de medidas através de comunicação sem
fio\cite{Clicia}.
\end{block}

\begin{block}

A resposta produzida pode ser mensurada em relação às mudanças físicas observadas, como temperatura, umidade, quantidade de luz etc.
\end{block}

\end{frame}

\begin{frame}
\label{slide_12}
\frametitle{Características}
\framesubtitle{Elementos de um sensor}

\begin{block}

Elementos de um sensor típico: \pause

\begin{itemize}

\item \textbf{Bateria:} onde está armazenada a energia do sensor. Possui uma taxa de consumo e uma capacidade finita \pause

\item \textbf{Transceptor:} sistema de transmissão e recepção. Consome energia de acordo com a operação realizada (geralmente transmitir é mais custoso que receber) \pause

\item \textbf{Processador:} unidade de processamento central do sensor. O consumo de energia depende da frequência e do modo de operação \pause

\item \textbf{Sensores:} dispositivos de aquisição de dados do fenômeno. O consumo está relacionado ao modo de operação e ao que está sendo medido pelos sensores (grandeza) 

\end{itemize}

\end{block}

\end{frame}

\begin{frame}
\label{slide_13}
\frametitle{Características}
\framesubtitle{Observador}

\begin{block}

Usuário final interessado em obter as informações enviadas pela rede de sensores relativas a um fenômeno. 
\end{block}

\begin{block}

O usuário pode indicar interesses (ou consultas) para a rede e receber respostas a estas consultas. 
\end{block}

\begin{block}

Podem existir, simultaneamente, múltiplos observadores numa rede de sensores.
\end{block}

\end{frame}

\begin{frame}
\label{slide_14}
\frametitle{Características}
\framesubtitle{Fenômeno}

\begin{block}

Entidade de interesse do observador que é monitorada e cuja
informação será analisada/filtrada pela rede de sensores.
\end{block} \pause

\begin{block}

Múltiplos fenômenos podem ser observados concomitantemente numa rede.

\end{block}

\end{frame}

\begin{frame}
\label{slide_15}
\frametitle{Características}
\framesubtitle{Endereçamento}

\begin{block}

Cada sensor de uma rede de sensores sem fio pode ser acessado individualmente ou não, dependendo da aplicação. 

\end{block} \pause

\begin{exampleblock}

Exemplo: sensores que monitoram funções do corpo humano.

\end{exampleblock}


\end{frame}

\begin{frame}
\label{slide_16}
\frametitle{Características}
\framesubtitle{Agregação de dados}

\begin{block}

Uma rede de sensores com função de agregação de dados pode agregar (ou sumarizar) os dados obtidos por diferentes nós antes do envio à estação base\footnote{Ponto de recepção das mensagens dos sensores da rede.}, reduzindo o tráfego de mensagens.

\end{block}

\end{frame}

\begin{frame}
\label{slide_17}
\frametitle{Características}
\framesubtitle{Mobilidade}

\begin{block}

Em uma rede de sensores sem fio, os sensores devem ter a capacidade de se adaptarem para continuar de acordo com os interesses do observador caso haja mobilidade.
\end{block} \pause

\begin{exampleblock}

Exemplos: \pause

\begin{itemize}
\item Nós estáticos utilizados para monitoramento do ritmo cardíaco de um paciente \pause
\item Nós móveis utilizados para monitorar condições dentro de um tornado
\end{itemize}

\end{exampleblock}

\end{frame}

\begin{frame}
\label{slide_18}
\frametitle{Características}
\framesubtitle{Número de sensores}

\begin{block}

Característica que requer atenção especial, pois o número de sensores não deve interferir na eficiência da rede (escalabilidade).
\end{block}

\end{frame}

\begin{frame}
\label{slide_19}
\frametitle{Características}
\framesubtitle{Limitação de consumo de energia}

\begin{block}

Redes de sensores sem fio são extremamente sensíveis ao consumo de energia. 
\end{block} \pause
\begin{block}

O uso de algoritmos, protocolos e aplicações mais robustos e eficientes pode não ser a melhor opção para este tipo de rede. 

\end{block} \pause

\begin{block}

Deve-se levar em consideração o acesso para manutenção dos sensores, que podem estar em áreas remotas.

\end{block} \pause

\begin{block}

A durabilidade de sua fonte de alimentação determina sua vida útil.

\end{block}

\end{frame}

\begin{frame}
\label{slide_20}
\frametitle{Características}
\framesubtitle{Auto-organização}

\begin{block}

Sensores podem ficar inacessíveis por problemas físicos, como falta de energia, problemas no canal de comunicação sem fio ou ainda por decisão de algum algoritmo de gerenciamento da rede (economia de energia, já existe outro sensor na região que está coletando o dado desejado etc). 

\end{block} \pause

\begin{block}

Pode ocorrer o contrário: sensores inativos passarem à atividade ou a inserção de novos sensores na rede. 
\end{block} \pause

\begin{block}

Em ambos os casos, é necessário que existam ferramentas para a auto-organização da rede, de modo que esta cumpra seus objetivos.

\end{block} \pause

\begin{alertblock}

Essa configuração deve ser feita periodicamente e precisa ser automática, já que o processo manual é totalmente inviável devido a problemas de acesso e escalabilidade. 
\end{alertblock}

\end{frame}

\begin{frame}
\label{slide_21}
\frametitle{Características}
\framesubtitle{Resposta a consultas}

\begin{block}

Uma consulta pode ser solicitada a um nó (sensor) individual ou a um grupo de nós.

\end{block} \pause

\begin{block}

Dependendo do processo de sumarização, pode não ser viável a transmissão dos dados até o nó sorvedouro\footnote{Ponto que gera os interesses iniciais e recebe os dados de interesse} através da rede. 
\end{block} \pause

\begin{block}

Pode ser necessário definir muitos nós sorvedouros que coletarão os dados de uma determinada área, respondendo às consultas referentes aos sensores sob sua responsabilidade.
\end{block} 

\end{frame}

\section{Métricas de desempenho}

\begin{frame}
\label{slide_22}
\frametitle{Métricas de desempenho}
\framesubtitle{Eficiência de energia e vida útil}

\begin{block}

Devido aos sensores possuírem baterias como fonte de energia, é necessário que os protocolos sejam eficientes em relação ao uso de energia, fazendo com que a vida útil do sistema possa ser aumentada. 
\end{block} \pause

\begin{alertblock}

Isso torna a conservação de energia um dos tópicos mais importantes a serem considerados no projeto de uma rede de sensores sem fio.

\end{alertblock}

\end{frame}

\begin{frame}
\label{slide_23}
\frametitle{Métricas de desempenho}
\framesubtitle{Eficiência de energia e vida útil}

\begin{block}

Na comunicação, a maior parte do consumo de energia está na transmissão e na recepção de dados. 

\end{block} \pause

\begin{block}

No processamento, sempre que possível, devem ser empregados métodos de economia de energia nas CPUs dos nós sensores.

\end{block}

\end{frame}

\begin{frame}
\label{slide_24}
\frametitle{Métricas de desempenho}
\framesubtitle{Energias alternativas}

\begin{block}

Pode-se aumentar o tempo de vida útil de um sensor aproveitando algum tipo de energia presente no ambiente, como eólica, solar etc. 

\end{block}

\end{frame}

\begin{frame}
\label{slide_25}
\frametitle{Métricas de desempenho}
\framesubtitle{Latência e precisão}

\begin{block}

Dependendo do fenômeno em observação, pode ser necessário analisá-lo em um certo espaço de tempo (latência), obtendo informações precisas e confiáveis. 
\end{block} \pause

\begin{block}

O sistema precisa ser eficiente e eficaz.
\end{block} \pause

\begin{block}

A rede de sensores deve ser estruturada de maneira a obter a precisão e a latência que satisfazem o observador, buscando sempre o uso mínimo de energia. 

\end{block}

\end{frame}

\begin{frame}
\label{slide_26}
\frametitle{Métricas de desempenho}
\framesubtitle{Tolerância a falhas}

\begin{block}

Sensores podem ficar inacessíveis por problemas físicos como a falta de energia, problemas no canal de comunicação sem fio ou por decisão de algum algoritmo de gerenciamento da rede.
\end{block} \pause

\begin{alertblock}

A rede de sensores sem fio deve ser robusta e sobreviver mesmo com a ocorrência de falhas em nós individuais, na rede ou que ocasionem conectividade intermitente. 
\end{alertblock} \pause

\begin{alertblock}

Falhas não catastróficas devem ser transparentes para a aplicação. 

\end{alertblock} \pause

\begin{alertblock}

A falha deve ser tratada como um acontecimento normal, e não como exceção.

\end{alertblock} \pause

\begin{block}

Pode-se alcançar a tolerância a falhas através da replicação de dados, porém esta operação requer energia. 
\end{block} 

\end{frame}

\begin{frame}
\label{slide_27}
\frametitle{Métricas de desempenho}
\framesubtitle{Escalabilidade}

\begin{block}

Redes de sensores sem fio podem possuir um grande número de nós, o que traz um desafio de escalabilidade. 
\end{block} \pause

\begin{alertblock}

Transmissão de dados redundantes e colisões provocam um gasto de energia desnecessário.  
\end{alertblock} \pause

\begin{alertblock}

O número de nós sensores presentes na rede não deve influenciar o seu desempenho. 
\end{alertblock} \pause

\begin{block}

Escalabilidade exige protocolos de roteamento, endereçamento e agregação de dados escaláveis.

\end{block}

\end{frame}

\begin{frame}
\label{slide_28}
\frametitle{Métricas de desempenho}
\framesubtitle{Exposição dos sensores}

\begin{block}

Medida da capacidade da rede em observar um certo objeto, movendo-se em um caminho arbitrário, em um determinado período de tempo.
 
\end{block} 

\end{frame}

\section{Arquitetura}
\begin{frame}
\label{slide_29}
\frametitle{Arquitetura}

\begin{block}

 \center \textbf{Arquitetura}
 
\end{block}

\end{frame}

\begin{frame}
\label{slide_30}
\frametitle{Arquitetura}

\begin{block}

\begin{itemize}

\item Infra-estrutura \pause

\item Aplicação \pause

\item Qualidade do serviço (QoS) \pause

\item Camadas \pause

\item Protocolos 

\end{itemize}

\end{block}

\end{frame}

\begin{frame}
\label{slide_31}
\frametitle{Arquitetura}
\framesubtitle{Infra-estrutura}

\begin{block}

A infra-estrutura é determinada pelas características dos sensores e a forma de utilizá-los.

\end{block}

\end{frame}

\begin{frame}
\label{slide_32}
\frametitle{Arquitetura}
\framesubtitle{Infra-estrutura - Características dos sensores}

\begin{block}

\begin{itemize}

\item Tamanho de memória \pause
\item Precisão na leitura \pause
\item Alcance de transmissão \pause
\item Vida útil da bateria 

\end{itemize}

\end{block}

\end{frame}

\begin{frame}
\label{slide_33}
\frametitle{Arquitetura}
\framesubtitle{Infra-estrutura - Características dos sensores}

\begin{block}

A precisão de um sensor depende de sua localização geográfica, do tamanho de seu \textit{buffer} (memória) e do tempo de processamento dos pacotes. 

\end{block} \pause

\begin{alertblock}

Múltiplos sensores podem gerar informações redundantes ou com um nível de precisão maior do que o necessário para a aplicação. 

\end{alertblock}

\end{frame}

\begin{frame}
\label{slide_34}
\frametitle{Arquitetura}
\framesubtitle{Infra-estrutura - Forma de utilização}

\begin{block}

\begin{itemize}

\item Número de sensores \pause
\item Localização dos sensores \pause
\item Nível de mobilidade dos sensores

\end{itemize}

\end{block}

\end{frame}

\begin{frame}
\label{slide_35}
\frametitle{Arquitetura}
\framesubtitle{Infra-estrutura - Forma de utilização}

\begin{block}

Aumentar o número de sensores pode gerar a falsa impressão de maior precisão na coleta de dados, caminhos mais eficientes e maior disponibilidade de energia na rede. 
\end{block} \pause

\begin{alertblock}

Se a capacidade do meio compartilhado é excedida (muitos sensores transmitindo ao mesmo tempo), ocorre congestionamento na rede e o desempenho como um todo cai.

\end{alertblock} \pause

\begin{alertblock}

Aumentar o número de sensores não garante qualidade de serviço em redes de sensores sem fio.

\end{alertblock}

\end{frame}

\begin{frame}
\label{slide_36}
\frametitle{Arquitetura}
\framesubtitle{Aplicação}

\begin{block}

Interface pela qual o observador faz consultas sobre o fenômeno.

\end{block} \pause

\begin{block}

A aplicação transforma os dados enviados pelos sensores em informações para o observador.

\end{block} \pause

\begin{block}

As consultas podem ser \textbf{estáticas} ou \textbf{dinâmicas}. 

\end{block}

\end{frame}

\begin{frame}
\label{slide_37}
\frametitle{Arquitetura}
\framesubtitle{Qualidade do serviço - QoS}

\begin{block}

Uma rede de sensores sem fio deve ter uma infra-estrutura e protocolos capazes de garantir a qualidade do serviço: \pause

\begin{itemize}

\item Precisão \pause
\item Latência \pause
\item Tolerância a falhas \pause
\item Energia 

\end{itemize}

\end{block}

\end{frame}

\begin{frame}
\label{slide_38}
\frametitle{Arquitetura}
\framesubtitle{Camadas}

\begin{block}

As camadas permitem o isolamento de tarefas.

\end{block} \pause

\begin{block}

É correto afirmar que uma determinada aplicação de um sensor pode ser definida independentemente do meio de transmissão utilizado.

\end{block} \pause

\begin{block}

Em uma rede de sensores, têm-se as camadas: \pause

\begin{itemize}

\item Física \pause
\item Enlace \pause
\item Rede \pause
\item Aplicação

\end{itemize}

\end{block}

\end{frame}

\begin{frame}
\label{slide_39}
\frametitle{Arquitetura}
\framesubtitle{Camadas - Física}

\begin{block}

A tarefa da camada física é a transmissão de mensagens entre sensores.

\end{block} \pause

\begin{block}

Responsável por selecionar as frequências que serão utilizadas, gerar a portadora, detectar, modular e codificar o sinal. 

\end{block} \pause

\begin{block}

A comunicação através de sinais de RF (rádio frequência) é a mais comum. 
\end{block} \pause

\begin{block}

A principal vantagem é que, ao contrário do que ocorre nas comunicações ópticas ou por infravermelho, neste tipo de comunicação o transmissor e o receptor não precisam estar alinhados.

\end{block}

\end{frame}

\begin{frame}
\label{slide_40}
\frametitle{Arquitetura}
\framesubtitle{Camadas - Enlace}

\begin{block}

Como as redes de sensores sem fio não exigem a definição prévia de uma infra-estrutura, os sensores devem possuir algum mecanismo que permita a identificação dos demais sensores na rede. 
\end{block} \pause

\begin{block}

Esta tarefa é realizada pela camada de enlace.

\end{block} \pause

\begin{block}

Além do controle de acesso ao meio (MAC), esta camada realiza as tarefas de controle de erros, detecção de quadros e multiplexação do fluxo de dados.

\end{block} \pause

\begin{block}

Como os sensores têm liberdade para deslocarem-se na rede, o controle de acesso ao meio é responsável pelo estabelecimento da comunicação \textit{multihop}, como forma de organizar a rede e estabelecer rotas.

\end{block}

\end{frame}

\begin{frame}
\label{slide_41}
\frametitle{Arquitetura}
\framesubtitle{Camadas - Rede}

\begin{block}

A camada de rede é responsável pelo roteamento de dados entre os sensores. 

\end{block} \pause

\begin{alertblock}

Os protocolos de roteamento utilizados devem suportar a comunicação \textit{multihop} e buscar sempre o uso mais eficiente possível da energia do sensor.

\end{alertblock}

\end{frame}

\begin{frame}
\label{slide_42}
\frametitle{Arquitetura}
\framesubtitle{Camadas - Aplicaçao}

\begin{block}

As aplicações de uma rede de sensores sem fio variam para cada caso.

\end{block} \pause

\begin{block}

Entre os protocolos atualmente definidos para a camada de aplicação, destacam-se: \pause

\begin{itemize}

\item \textit{Sensor Management Protocol} (SMP) (gerenciamento e agrupamento dos sensores) \pause
\item \textit{Task Assignment and Data Advertisement Protocol}   (TADAP) (interesses do observador) \pause
\item \textit{Sensor Query and Data Dissemination Protocol} (SQDDP) (interface para consultas aos sensores)

\end{itemize}

\end{block}

\end{frame}

\begin{frame}
\label{slide_43}
\frametitle{Arquitetura}
\framesubtitle{Protocolos}

\begin{block}

Responsável pela comunicação entre os sensores e entre sensores e observadores. 
\end{block} \pause

\begin{block}

A maioria das aplicações usa protocolos MAC (\textit{Medium Access Control}) em comunicação sem fio. 

\end{block}

\end{frame}

\begin{frame}
\label{slide_44}
\frametitle{Arquitetura}
\framesubtitle{Protocolos}

\begin{block}

Os protocolos de roteamento \textit{ad hoc} podem ser
usados para redes de sensores.

\end{block} \pause

\begin{alertblock}

Contudo, estes protocolos apresentam desvantagens: \pause

\begin{itemize}

\item Sensores têm pouca carga de bateria e memória restrita \pause

\item A tabela de roteamento cresce com o tamanho da rede \pause

\item Redes de sensores redes são projetadas para comunicação fim a fim e estes protocolos não reagem adequadamente caso ocorra movimentação \pause

\item Requisições de endereçamento podem não ser
apropriadas para redes de sensores \pause

\item Protocolos de roteamento para redes \textit{ad hoc} não suportam disseminação cooperativa (fusão ou agregação de dados). 

\end{itemize}

\end{alertblock}

\end{frame}

\begin{frame}
\label{slide_45}
\frametitle{Arquitetura}
\framesubtitle{Protocolos}

\begin{block}

Uma antiga técnica de roteamento para redes de sensores é o \textit{flooding} (inundação), baseada em \textit{broadcast}. 

\end{block} \pause

\begin{block}

Os sensores propagam sua informação para todos seus vizinhos, em \textit{broadcast}, e esses vizinhos fazem a mesma coisa com a informação até que esta atinja o sorvedouro.

\end{block} \pause

\begin{alertblock}

Entretanto...: \pause

\begin{itemize}

\item Pode causar um alto \textit{overhead} \pause

\item Pode ocorrer implosão (um nó recebe a mesma mensagem por dois - ou mais - vizinhos diferentes) \pause

\item Pode ocorrer a superposição (dois sensores, que atuam em um mesmo campo de observação, acabam detectando uma mesma situação e gerando uma mesma mensagem, propagando-a, ambos, para um vizinho em comum)

\end{itemize}

\end{alertblock}

\end{frame}

\begin{frame}
\label{slide_46}
\frametitle{Arquitetura}
\framesubtitle{Protocolos}

\begin{block}

A técnica conhecida como \textit{gossiping} (“fofoca”) é uma derivação do \textit{flooding}, em que, ao invés de fazer \textit{broadcast} da mensagem, o sensor a transmite para um sensor vizinho escolhido aleatoriamente, e assim por diante, até chegar no sorvedouro.

\end{block} \pause

\begin{alertblock}

Essa técnica evita a implosão, mas leva-se muito tempo para que a informação percorra a rede.  

\end{alertblock}

\end{frame}

\begin{frame}
\label{slide_47}
\frametitle{Arquitetura}
\framesubtitle{Protocolos - Outros protocolos}

\begin{block}

Protocolos para redes planas: \pause

\begin{itemize}

\item \textit{Directed Diffusion} \pause
\item SPIN (\textit{Sensor Protocol for Information via Negotiation}) \pause
\item SAR (\textit{Sequential Assignment Routing}) \pause
\item \textit{Adaptive Local Routing Cooperative Signal Processing: Noncoherent Processing e Coherent Processing}

\end{itemize}

\end{block}

\end{frame}

\begin{frame}
\label{slide_48}
\frametitle{Arquitetura}
\framesubtitle{Protocolos - Outros protocolos}

\begin{block}

Protocolos para redes hierárquicas: \pause

\begin{itemize}

\item LEACH (\textit{Low Energy Adaptive Clustering Hierarchy}) \pause
\item CBRP (\textit{Cluster Based Routing Protocol}) \pause
\item TEEN (\textit{Threshold-sensitive Energy Efficient Network}) \pause
\item APTEEN (\textit{Adaptive Periodic Threshold-sensitive Energy Efficient Network}) \pause
\item PEGASIS (\textit{Power Efficient Gathering in Sensor Information System})

\end{itemize}

\end{block}

\end{frame}

\begin{frame}
\label{slide_49}
\frametitle{Arquitetura}
\framesubtitle{Protocolos - \textit{Directed Diffusion}}

\begin{block}

Na difusão direcionada, o sensor que deve transmitir nomeia os dados usando atributos que descrevem a tarefa a ser desempenhada.

\end{block} \pause

\begin{block}

A estação base (sorvedouro) propaga seus interesses, ou seja, quais atributos quer receber. 
\end{block} \pause

\begin{block}

Os sensores vizinhos propagam essa informação que, ao passar pelos sensores, informam a distância percorrida e, ao chegar em um sensor que contenha o atributo de interesse, este o envia através do caminho informado até a estação base.

\end{block} \pause

\begin{block}

Não há identificação prévia dos sensores na rede. 

\end{block} 

\end{frame}

\begin{frame}
\label{slide_50}
\frametitle{Arquitetura}
\framesubtitle{Protocolos - SAR (\textit{Sequential Assignment Routing})}

\begin{block}

O SAR realiza roteamento \textit{multihop}, utilizando tabelas, pelas quais faz uma seleção de múltiplos caminhos, evitando \textit{overhead} em caso de falha.

\end{block} \pause

\begin{block}

Cria diversas árvores, sendo que a raiz de cada uma delas é vizinha à estação base. 

\end{block} \pause

\begin{block}

Baseia a escolha do caminho a ser utilizado de acordo com os recursos de energia disponíveis, QoS e a prioridade do pacote a ser enviado, através de uma métrica ponderada entre estes fatores.

\end{block} \pause

\begin{block}

As árvores, que formam os múltiplos caminhos, evitam os nós problemáticos.

\end{block} \pause

\begin{block}

Um sensor pode pertencer a várias árvores distintas e enviar suas mensagens por uma entre as várias árvores disponíveis.

\end{block}

\end{frame}

\begin{frame}
\label{slide_51}
\frametitle{Arquitetura}
\framesubtitle{Protocolos - SPIN (\textit{Sensor Protocol for Information via Negotiation})}

\begin{block}

O SPIN é, na verdade, uma família de protocolos adaptativos para redes de sensores.

\end{block} \pause

\begin{block}

Nos protocolos SPIN, os sensores utilizam descritores, chamados de \textbf{meta-dados}, para nomear seus dados.

\end{block} \pause

\begin{block}

Ao invés de difundirem os dados pela rede, difundem os meta-dados, que são de tamanho menor, atendendo ao problema de escassez de energia, em comparação ao \textit{flooding}.

\end{block} \pause

\begin{block}

Possuem gerenciamento de recursos, que permite tomar decisões de forma a não gastar uma quantidade de energia que possa levar ao desligamento do sensor.

\end{block}

\end{frame}

\begin{frame}
\label{slide_52}
\frametitle{Arquitetura}
\framesubtitle{Protocolos - LEACH (\textit{Low Energy Adaptive Clustering Hierarchy})}

\begin{block}

Protocolo eficiente em energia para redes sem mobilidade.

\end{block} \pause

\begin{block}

Utiliza uma arquitetura baseada em \textit{clusters}, onde os sensores que fazem parte de um \textit{cluster} enviam seus dados apenas para o sensor raiz desse \textit{cluters}, o \textit{cluster-head}. 

\end{block} \pause

\begin{block}

No \textit{cluster-head} há a agregação dos dados de cada sensor, com o tratamento de informações redundantes e o envio desses dados para a estação base.

\end{block} \pause

\begin{block}

Há uma rotação dos \textit{cluster-heads}, proporcionando um gasto de energia mais uniforme entre os nós e evitando que a perda de um \textit{cluster-head} leve à inutilização da rede.

\end{block} \pause

\begin{block}

A comunicação entre os nós e o \textit{cluster-head} é feita através de TDMA.

\end{block}

\end{frame}

\section{Modelos}

\begin{frame}
\label{slide_53}
\frametitle{Modelos}

\begin{block}

 \center \textbf{Modelos}

\end{block}

\end{frame}

\begin{frame}
\label{slide_54}
\frametitle{Modelos}
\framesubtitle{Modelos de comunicação}

\begin{block}

O protocolo de uma rede de sensores sem fio deve permitir dois modelos de comunicação: \pause

\begin{itemize}

\item Aplicação \pause
\item Infra-estrutura

\end{itemize}

\end{block}

\end{frame}

\begin{frame}
\label{slide_55}
\frametitle{Modelos}
\framesubtitle{Comunicação de aplicação}

\begin{block}

Envio dos dados coletados pelos sensores para a aplicação.

\end{block} \pause

\begin{block}

Os dados são tratados a fim de informar o observador sobre o fenômeno.

\end{block} \pause

\begin{block}

Pode ocorrer de duas formas: \pause

\begin{itemize}

\item Cooperativa: um nó transmite seus dados para um nó próximo e este manda para um outro próximo e assim por diante até chegar a aplicação. \pause
\item Não-Cooperativa: os sensores atendem o observador sem se comunicarem entre si. 

\end{itemize}

\end{block}

\end{frame}

\begin{frame}
\label{slide_56}
\frametitle{Modelos}
\framesubtitle{Comunicação de infra-estrutura}

\begin{block}

Necessária para configurar, manter e otimizar a rede.

\end{block} \pause

\begin{block}

Os sensores têm que ser capazes de encontrar caminhos dinâmicos para outros sensores, para o observador e para o fenômeno.

\end{block} \pause

\begin{block}

Pode gerar \textit{overhead}, portanto é importante minimizá-la, para evitar que o requisito de latêcia não seja atendido. 

\end{block} \pause

\begin{block}

Na fase inicial da configuração de uma rede de sensores pode haver um consumo maior de energia, assim como quando os sensores são móveis. 

\end{block}

\end{frame}

\begin{frame}
\label{slide_57}
\frametitle{Modelos}
\framesubtitle{Modelos de envio de dados}

\begin{block}

\begin{itemize}

\item \textbf{Modelo Contínuo} – os sensores enviam os seus dados com uma taxa pré-especificada e fixa. \pause
\item \textbf{Modelo de Dados Orientado a Eventos} – os sensores apenas enviam dados para o observador se um determinado evento de interesse pré-determinado ocorrer. \pause
\item \textbf{Modelo Iniciado pelo Observador} – o observador precisa fazer uma requisição explícita para que os sensores enviem dados. 

\end{itemize}
\end{block}

\end{frame}

\begin{frame}
\label{slide_58}
\frametitle{Modelos}
\framesubtitle{Modelos de rede}

\begin{block}

\begin{itemize}

\item Redes Estáticas \pause
\item Redes Dinâmicas

\end{itemize}

\end{block}

\end{frame}

\begin{frame}
\label{slide_59}
\frametitle{Modelos}
\framesubtitle{Modelos de rede - Redes Estáticas}

\begin{block}

Os sensores, o observador e o fenômeno não se movem.

\end{block} \pause

\begin{block}

Um nó eleito transmite um resumo das características locais para o observador podendo haver vários níveis hierárquicos. 

\end{block} \pause

\begin{block}

Após uma comunicação de infra-estrutura inicial para criar um caminho entre o sensor e o observador, apenas ocorre comunicação de aplicação. 

\end{block}

\end{frame}

\begin{frame}
\label{slide_60}
\frametitle{Modelos}
\framesubtitle{Modelos de rede - Redes Dinâmicas}

\begin{block}

Os sensores, o observador ou o fenômeno podem se mover.

\end{block} \pause

\begin{block}

Durante o tempo de configuração inicial, o observador pode criar vários caminhos entre ele e o fenômeno e colocá-los em cache para escolher o melhor caminho para momentos diferentes.  

\end{block}

\end{frame}

\begin{frame}
\label{slide_61}
\frametitle{Modelos}
\framesubtitle{Modelos de rede - Redes Dinâmicas}

\begin{block}

Caso não haja um caminho válido no cache, duas estratégias podem ser usadas: \pause

\begin{itemize}

\item \textbf{Reativa}, onde o observador procura recuperar o caminho apenas depois de observar um caminho com falha. \pause

\item \textbf{Pró-ativa}, onde um sensor que faz parte do caminho lógico entre o observador e o fenômeno pretende deixar esse caminho. Para fazê-lo é necessário procurar um sensor livre e disposto a substituí-lo. No caso de nenhum sensor livre ser encontrado, uma mensagem de invalidação de caminho é enviada para o observador. 

\end{itemize}

\end{block}

\end{frame}

\section{Segurança}

\begin{frame}
\label{slide_62}
\frametitle{Segurança}

\begin{block}

 \center \textbf{Segurança}

\end{block}

\end{frame}

\begin{frame}
\label{slide_63}
\frametitle{Segurança}
\framesubtitle{Requisitos}

\begin{block}

Requisitos básicos de segurança: \pause

\begin{itemize}

\item Confiabilidade \pause
\item Disponibilidade

\end{itemize}

\end{block} 

\end{frame}

\begin{frame}
\label{slide_64}
\frametitle{Segurança}
\framesubtitle{Requisitos básicos}

\begin{block}

\textbf{Confiabilidade:} capacidade de um sistema de responder a uma determinada especificação seguindo condições previamente definidas em um dado período de tempo.

\end{block} \pause

\begin{block}

\textbf{Disponibilidade:} probabilidade do sistema estar funcionando em um determinado instante.

\end{block}

\end{frame}

\begin{frame}
\label{slide_65}
\frametitle{Segurança}
\framesubtitle{Requisitos específicos}

\begin{block}

Requisitos específicos para uma rede de sensores sem fio: \pause

\begin{itemize}

\item Confidencialidade dos dados \pause
\item Autenticação dos dados \pause
\item Integridade dos dados \pause
\item Atualidade dos dados

\end{itemize}

\end{block} 

\end{frame}

\begin{frame}
\label{slide_66}
\frametitle{Segurança}
\framesubtitle{Requisitos específicos - Confidencialidade dos dados}

\begin{block}

Garante que as informações em uma rede de sensores não sejam acessadas por pessoas ou programas não autorizados. 
\end{block} \pause

\begin{block}

O método padrão para manter os dados secretos é a criptografia dos dados com uma chave secreta.

\end{block} 

\end{frame}

\begin{frame}
\label{slide_67}
\frametitle{Segurança}
\framesubtitle{Requisitos específicos - Autenticação dos dados}

\begin{block}

O receptor deve estar seguro que os dados são oriundos da fonte correta, ou seja, deve-se identificar a autoria de uma determinada ação ou transmissão. 

\end{block} \pause

\begin{block}

A autenticação pode ser garantida por meio de um mecanismo simétrico, de modo que emissor e receptor compartilham de uma chave secreta que serve para a geração de um código de autenticação de mensagem (MAC - \textit{Message Authentication Code}) para todo dado comunicado.

\end{block} 

\end{frame}

\begin{frame}
\label{slide_68}
\frametitle{Segurança}
\framesubtitle{Requisitos específicos - Integridade dos dados}

\begin{block}

Garante ao receptor que não houve alterações (intencionais ou não) nos dados recebidos durante o seu trânsito. 

\end{block} 

\end{frame}

\begin{frame}
\label{slide_69}
\frametitle{Segurança}
\framesubtitle{Requisitos específicos - Atualidade dos dados}

\begin{block}

Certifica que não houve interferência de mensagens antigas.

\end{block} \pause

\begin{block}

Pode ser garantido se for feita a ordenação parcial das mensagens, sem causar atraso da informação (usado para a medida de sensores) ou a ordem total de um par requisição-resposta, permitindo estimar o atraso (usado para a sincronização de tempo na rede).  

\end{block} 

\end{frame}

\section{Considerações finais}
\begin{frame}
\label{slide_70}
\frametitle{Considerações finais}

\begin{block}

 \center \textbf{Considerações finais}

\end{block}

\end{frame}

\begin{frame}
\label{slide_71}
\frametitle{Considerações finais}

\begin{block}

Apesar de apresentarem características comuns às redes móveis \textit{ad hoc}, as redes de sensores sem fio não podem ser abordadas e tratadas como tais, pois neste tipo de rede os nós têm baixa capacidade de energia e disponibilidade de memória.
\end{block} \pause

\begin{block}

Desta forma, os protocolos de roteamento utilizados para redes \textit{ad hoc} não são apropriados, por gerarem grandes tabelas de roteamento – memória insuficiente nos nós sensores –, além de não apresentarem suporte a agregação de dados e à criação e manutenção de rotas – importante quando se trata da energia dos nós.  

\end{block} 

\end{frame}

\begin{frame}
\label{slide_72}
\frametitle{Considerações finais}
\framesubtitle{Conceitos e desafios}

\begin{block}

As redes de sensores sem fio trazem novos conceitos e alguns desafios: \pause

\begin{itemize}


\item Capacidade de auto-organização \pause
\item Topologia dinâmica \pause
\item Pouca disponibilidade de energia \pause
\item Fornecimento de informações atuais e corretas do fenômeno \pause

\end{itemize}

\end{block}

\begin{exampleblock} 

\begin{itemize}

\item Novas oportunidades de pesquisa \pause
\item Base para o desenvolvimento e concretização da computação ubíqua.

\end{itemize}

\end{exampleblock}

\end{frame}

\section{Referências}
\begin{frame}
\label{slide_73}
\frametitle{Referências}

\bibliographystyle{acm}
\bibliography{ref}

\end{frame}

%%%%% Thanks page
\begin{frame}
\label{slide_74}
\frametitle{Agradecimentos}
\vskip20pt

\begin{center}
{\bf \color{alert} Obrigado pela atenção!}
\end{center}

\vskip20pt

\begin{center}
Apresentação disponível em nosso grupo de \textit{e-mails}:\\
%\url{http:/}
\vskip12pt
\end{center}

\titlepage
\end{frame}

\end{document}