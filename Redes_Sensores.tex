\documentclass[notes]{beamer}
\usepackage{graphicx}
\usepackage{url}

\mode<presentation>
{
  % A tip: pick a theme you like first, and THEN modify the color theme, and then add math content.
  % Warsaw is the theme selected by default in Beamer's installation sample files.

  %%%%%%%%%%%%%%%%%%%%%%%%%%%% THEME
%\usetheme{AnnArbor}
%\usetheme{Antibes}
%\usetheme{Bergen}
%\usetheme{Berkeley}
%\usetheme{Berlin}
%   \usetheme{Boadilla}
%\usetheme{boxes}
%\usetheme{CambridgeUS}
%   \usetheme{Copenhagen}
%   \usetheme{Darmstadt}
%\usetheme{default}
%\usetheme{Dresden}
   \usetheme{Frankfurt}
%\usetheme{Goettingen}
%\usetheme{Hannover}
%\usetheme{Ilmenau}
%\usetheme{JuanLesPins}
%\usetheme{Luebeck}
%\usetheme{Madrid}
%\usetheme{Malmoe}
%\usetheme{Marburg}
%\usetheme{Montpellier}
%\usetheme{PaloAlto}
%\usetheme{Pittsburgh}
%\usetheme{Rochester}
%\usetheme{Singapore}
%\usetheme{Szeged}
%\usetheme{Warsaw}

  %%%%%%%%%%%%%%%%%%%%%%%%%%%% COLOR THEME
  %\usecolortheme{albatross}
  %\usecolortheme{beetle}
  %\usecolortheme{crane}
  %\usecolortheme{default}
  %\usecolortheme{dolphin}
  %\usecolortheme{dove}
  %\usecolortheme{fly}
  %\usecolortheme{lily}
  \usecolortheme{orchid}
  %\usecolortheme{rose}
  %\usecolortheme{seagull}
  %\usecolortheme{seahorse}
  %\usecolortheme{sidebartab}
  %\usecolortheme{structure}
  %\usecolortheme{whale}

  %%%%%%%%%%%%%%%%%%%%%%%%%%%% OUTER THEME
  %\useoutertheme{default}
  %\useoutertheme{infolines}
  %\useoutertheme{miniframes}
  %\useoutertheme{shadow}
  %\useoutertheme{sidebar}
  %\useoutertheme{smoothbars}
  %\useoutertheme{smoothtree}
  %\useoutertheme{split}
  %\useoutertheme{tree}

  %%%%%%%%%%%%%%%%%%%%%%%%%%%% INNER THEME
  %\useinnertheme{circles}
  %\useinnertheme{default}
  %\useinnertheme{inmargin}
  %\useinnertheme{rectangles}
  %\useinnertheme{rounded}

  %%%%%%%%%%%%%%%%%%%%%%%%%%%%%%%%%%%

  \setbeamercovered{transparent} % or whatever (possibly just delete it)
  % To change behavior of \uncover from graying out to totally invisible, can change \setbeamercovered to invisible instead of transparent. apparently there are also 'dynamic' modes that make the amount of graying depend on how long it'll take until the thing is uncovered.

}


% Get rid of nav bar
\beamertemplatenavigationsymbolsempty

% Use short top
\usepackage[headheight=12pt,footheight=12pt]{beamerthemeboxes}
%\addheadboxtemplate{\color{black}}{
%\hskip0.3cm
%\color{white}
%\insertshortauthor \ \ \ \ 
%\insertframenumber \ \ \ \ \ \ \ 
%\insertsection \ \ \ \ \ \ \ \ \ \ \ \ \ \ \ \ \  \insertsubsection
%\hskip0.3cm}
%\addheadboxtemplate{\color{black}}{
%\color{white}
%\ \ \ \ 
%\insertsection
%}
%\addheadboxtemplate{\color{black}}{
%\color{white}
%\ \ \ \ 
%\insertsubsection
%}

% Insert frame number at bottom of the page.
\usefoottemplate{\hfil\tiny{\color{black!90}\insertframenumber}} 
\setbeamertemplate{footline}[frame number]
\usepackage[english, portuguese]{babel}
\usepackage[latin1, utf8]{inputenc}

\usepackage{times}
\usepackage[T1]{fontenc}

\title{Redes Sensores}
\subtitle{}

\author{Adriano Ricardo Ruggero}

\institute{Instituto de Computação - Unicamp}

\date{\today}

\subject{Talks}

\def\defn#1{{\color{red} #1}}

\begin{document}

\begin{frame}
  \titlepage
\end{frame}

\begin{frame}
  \frametitle{Agenda}
  \tableofcontents
\end{frame}

\section{Introdução}

\section{Características}

\section{Desempenho}

\section{Arquitetura}

\section{Modelos}

\section{Protocolos}

\section{Segurança}

\section{Considerações finais}

\begin{frame}

\frametitle{Introdução}

\framesubtitle{Definições:} 

\begin{block}

Uma rede sem fio formada por um grande número de sensores pequenos e imóveis  que detectam e transmitem alguma característica física do ambiente. A informação contida nos sensores é agregada numa base central de dados \cite{Malladi_1}

\end{block}

\end{frame}

\begin{frame}
\frametitle{Introdução}
\framesubtitle{Definições:} 

\begin{block}

Uma classe particular de sistemas distribuídos, onde as comunicações de baixo
nível não dependem da localização topológica da rede \cite{Huang_2}

\end{block}

\end{frame}

\begin{frame}
\frametitle{Introdução}
\framesubtitle{Definições:} 
\begin{block}

Um conjunto de nós individuais (sensores) que operam sozinhos, mas que podem formar uma rede com o objetivo de juntar as informações individuais de cada sensor para monitorar algum fenômeno.

\end{block}

\end{frame}

\begin{frame}
\frametitle{Introdução}
\framesubtitle{Objetivos}

\begin{block}

O objetivo principal é monitorar fenômenos em ambientes perigosos, de difícil acesso ou ambientes de interação direta com um indivíduo e transmiti-los em forma de dados. 
\end{block}

\begin{block}

Cada nó sensor gera e armazena dados independentemente dos outros.

\end{block}

\begin{block}

Apesar de menor e menos confiável do que equipamentos de rede tradicionais, juntos são capazes de monitorar fenômenos complexos. 

\end{block}

\end{frame}

\begin{frame}
\frametitle{Introdução}
\framesubtitle{Vantagens}

\begin{itemize}

\item O uso de tecnologias de rede já existentes confere um custo menor e melhor desempenho ao sistema; \pause

\item Permite o monitoramento de locais perigosos e de difícil acesso; \pause

\item A combinação de sensores com diferentes freqüências confere maior precisão às medidas coletadas (desde que o posicionamento dos sensores seja preciso e haja sincronização entre eles). 

\end{itemize}

\end{frame}

\begin{frame}
\frametitle{Introdução}
\framesubtitle{Usos}

\begin{block}

A área de Redes de Sensores sem Fio têm se mostrado um grande potencial em fazer parte da vida das pessoas no futuro, levando a uma “simbiose” cada vez maior entre a máquina e o homem. 

\end{block}

\begin{block}

Observa-se isto pela gama de áreas em que redes de sensores podem ser utilizadas: na medicina, no tráfego urbano, nas indústrias, no controle da poluição, no estudo e prevenção de desastres naturais, na automação doméstica, em ambientes inteligentes, entre outras. 

\end{block}

\end{frame}

\begin{frame}

\frametitle{Características}

\begin{itemize}

\item Sensor \pause

\item Observador \pause

\item Fenômeno

\end{itemize}

\end{frame}

\begin{frame}

\frametitle{Sensor}

Dispositivo que monitora fisicamente um fenômeno ambiental e gera
relatórios de medidas através de comunicação sem
fio. A resposta produzida pode ser mensurada em relação às mudanças físicas observadas, como temperatura, umidade, quantidade de luz etc.

\end{frame}

\begin{frame}

\frametitle{Observador}

Usuário final interessado em obter
as informações enviadas pela rede de sensores relativas a um fenômeno. O usuário pode indicar interesses (ou consultas) para a rede e receber respostas a estas
consultas. Podem existir, simultaneamente,
múltiplos observadores numa rede de sensores.

\end{frame}

\begin{frame}

\frametitle{Fenômeno}

Entidade de interesse do observador que é monitorada e cuja
informação será analisada/filtrada pela rede de sensores. Múltiplos fenômenos
podem ser observados concomitantemente numa rede.

\end{frame}

\begin{frame}

\frametitle{Referências}

\bibliographystyle{acm}
\bibliography{ref}

\end{frame}

%%%%% Thanks page
\begin{frame}
\frametitle{Agradecimentos}
\vskip20pt

\begin{center}
{\bf \color{alert} Obrigado pela atenção!}
\end{center}

\vskip20pt

\begin{center}
Apresentação disponível em:\\
\url{http:/}
\vskip12pt
\end{center}

\titlepage
\end{frame}

\end{document}