\documentclass[notes]{beamer}
\usepackage{graphicx}
\usepackage{url}

\mode<presentation>
{
  % A tip: pick a theme you like first, and THEN modify the color theme, and then add math content.
  % Warsaw is the theme selected by default in Beamer's installation sample files.

  %%%%%%%%%%%%%%%%%%%%%%%%%%%% THEME
%\usetheme{AnnArbor}
%\usetheme{Antibes}
%\usetheme{Bergen}
%\usetheme{Berkeley}
%\usetheme{Berlin}
%   \usetheme{Boadilla}
%\usetheme{boxes}
%\usetheme{CambridgeUS}
%   \usetheme{Copenhagen}
%   \usetheme{Darmstadt}
%\usetheme{default}
%\usetheme{Dresden}
   \usetheme{Frankfurt}
%\usetheme{Goettingen}
%\usetheme{Hannover}
%\usetheme{Ilmenau}
%\usetheme{JuanLesPins}
%\usetheme{Luebeck}
%\usetheme{Madrid}
%\usetheme{Malmoe}
%\usetheme{Marburg}
%\usetheme{Montpellier}
%\usetheme{PaloAlto}
%\usetheme{Pittsburgh}
%\usetheme{Rochester}
%\usetheme{Singapore}
%\usetheme{Szeged}
%\usetheme{Warsaw}

  %%%%%%%%%%%%%%%%%%%%%%%%%%%% COLOR THEME
  %\usecolortheme{albatross}
  %\usecolortheme{beetle}
  %\usecolortheme{crane}
  %\usecolortheme{default}
  %\usecolortheme{dolphin}
  %\usecolortheme{dove}
  %\usecolortheme{fly}
  %\usecolortheme{lily}
  \usecolortheme{orchid}
  %\usecolortheme{rose}
  %\usecolortheme{seagull}
  %\usecolortheme{seahorse}
  %\usecolortheme{sidebartab}
  %\usecolortheme{structure}
  %\usecolortheme{whale}

  %%%%%%%%%%%%%%%%%%%%%%%%%%%% OUTER THEME
  %\useoutertheme{default}
  %\useoutertheme{infolines}
  %\useoutertheme{miniframes}
  %\useoutertheme{shadow}
  %\useoutertheme{sidebar}
  %\useoutertheme{smoothbars}
  %\useoutertheme{smoothtree}
  %\useoutertheme{split}
  %\useoutertheme{tree}

  %%%%%%%%%%%%%%%%%%%%%%%%%%%% INNER THEME
  %\useinnertheme{circles}
  %\useinnertheme{default}
  %\useinnertheme{inmargin}
  %\useinnertheme{rectangles}
  %\useinnertheme{rounded}

  %%%%%%%%%%%%%%%%%%%%%%%%%%%%%%%%%%%

  \setbeamercovered{transparent} % or whatever (possibly just delete it)
  % To change behavior of \uncover from graying out to totally invisible, can change \setbeamercovered to invisible instead of transparent. apparently there are also 'dynamic' modes that make the amount of graying depend on how long it'll take until the thing is uncovered.

}


% Get rid of nav bar
\beamertemplatenavigationsymbolsempty

% Use short top
\usepackage[headheight=12pt,footheight=12pt]{beamerthemeboxes}
%\addheadboxtemplate{\color{black}}{
%\hskip0.3cm
%\color{white}
%\insertshortauthor \ \ \ \ 
%\insertframenumber \ \ \ \ \ \ \ 
%\insertsection \ \ \ \ \ \ \ \ \ \ \ \ \ \ \ \ \  \insertsubsection
%\hskip0.3cm}
%\addheadboxtemplate{\color{black}}{
%\color{white}
%\ \ \ \ 
%\insertsection
%}
%\addheadboxtemplate{\color{black}}{
%\color{white}
%\ \ \ \ 
%\insertsubsection
%}

% Insert frame number at bottom of the page.
\usefoottemplate{\hfil\tiny{\color{black!90}\insertframenumber}} 
\setbeamertemplate{footline}[frame number]
\usepackage[english, portuguese]{babel}
\usepackage[latin1, utf8]{inputenc}

\usepackage{times}
\usepackage[T1]{fontenc}

\title{Redes Sensores}
\subtitle{}

\author{Adriano Ricardo Ruggero}

\institute{Instituto de Computação - Unicamp}

\date{\today}

\subject{Talks}

\def\defn#1{{\color{red} #1}}

\begin{document}

\begin{frame}
  \titlepage
\end{frame}

\begin{frame}
  \frametitle{Agenda}
  \tableofcontents
\end{frame}

\section{Introdução}
\begin{frame}
\frametitle{Introdução}

\begin{block}

 \center \textbf{Introdução}
 
\end{block}

\end{frame}

\begin{frame}

\frametitle{Introdução}

\framesubtitle{Definições:} 

\begin{block}

Uma rede sem fio formada por um grande número de sensores pequenos e imóveis  que detectam e transmitem alguma característica física do ambiente. A informação contida nos sensores é agregada numa base central de dados \cite{Malladi_1}

\end{block}

\end{frame}

\begin{frame}
\frametitle{Introdução}
\framesubtitle{Definições:} 

\begin{block}

Uma classe particular de sistemas distribuídos, onde as comunicações de baixo
nível não dependem da localização topológica da rede \cite{Huang_2}

\end{block}

\end{frame}

\begin{frame}
\frametitle{Introdução}
\framesubtitle{Definições:} 
\begin{block}

Um conjunto de nós individuais (sensores) que operam sozinhos, mas que podem formar uma rede com o objetivo de juntar as informações individuais de cada sensor para monitorar algum fenômeno.

\end{block}

\end{frame}

\begin{frame}
\frametitle{Introdução}
\framesubtitle{Objetivos}

\begin{block}

O objetivo principal é monitorar fenômenos em ambientes perigosos, de difícil acesso ou ambientes de interação direta com um indivíduo e transmiti-los em forma de dados. 
\end{block} \pause

\begin{block}

Cada nó sensor gera e armazena dados independentemente dos outros.

\end{block} \pause

\begin{block}

Apesar de menor e menos confiável do que equipamentos de rede tradicionais, juntos são capazes de monitorar fenômenos complexos. 

\end{block}

\end{frame}

\begin{frame}
\frametitle{Introdução}
\framesubtitle{Vantagens}

\begin{block}

\begin{itemize}

\item O uso de tecnologias de rede já existentes confere um custo menor e melhor desempenho ao sistema; \pause

\item Permite o monitoramento de locais perigosos e de difícil acesso; \pause

\item A combinação de sensores com diferentes freqüências confere maior precisão às medidas coletadas (desde que o posicionamento dos sensores seja preciso e haja sincronização entre eles). 

\end{itemize}

\end{block}

\end{frame}

\begin{frame}
\frametitle{Introdução}
\framesubtitle{Usos}

\begin{block}

A área de Redes de Sensores sem Fio têm se mostrado um grande potencial em fazer parte da vida das pessoas no futuro, levando a uma “simbiose” cada vez maior entre a máquina e o homem. 

\end{block} \pause

\begin{block}

Observa-se isto pela gama de áreas em que redes de sensores podem ser utilizadas: na medicina, no tráfego urbano, nas indústrias, no controle da poluição, no estudo e prevenção de desastres naturais, na automação doméstica, em ambientes inteligentes, entre outras. 

\end{block}

\end{frame}

\section{Características}
\begin{frame}
\frametitle{Características}

\begin{block}

 \center \textbf{Características}
 
\end{block}

\end{frame}

\begin{frame}
\frametitle{Características}
\framesubtitle{Sensor}

\begin{block}

Dispositivo que monitora fisicamente um fenômeno ambiental e gera
relatórios de medidas através de comunicação sem
fio. A resposta produzida pode ser mensurada em relação às mudanças físicas observadas, como temperatura, umidade, quantidade de luz etc.
\end{block}

\end{frame}

\begin{frame}
\frametitle{Características}
\framesubtitle{Elementos de um sensor}

\begin{block}

Elementos de um sensor típico:

\begin{itemize}

\item \textbf{Bateria:} onde está armazenada a energia do sensor. Possui uma taxa de consumo e uma capacidade finita; \pause

\item \textbf{Transceptor:} sistema de transmissão e recepção. Consome energia de acordo com a operação realizada (geralmente transmitir é mais custoso que receber); \pause

\item \textbf{Processador:} unidade de processamento central do sensor. O consumo de energia depende da freqüência e do modo de operação; \pause

\item \textbf{Sensores:} dispositivos de aquisição de dados do fenômeno. O consumo está relacionado ao modo de operação e ao que está sendo medido pelos sensores (grandeza). 

\end{itemize}

\end{block}

\end{frame}

\begin{frame}
\frametitle{Características}
\framesubtitle{Observador}

\begin{block}

Usuário final interessado em obter
as informações enviadas pela rede de sensores relativas a um fenômeno. O usuário pode indicar interesses (ou consultas) para a rede e receber respostas a estas
consultas. Podem existir, simultaneamente,
múltiplos observadores numa rede de sensores.
\end{block}

\end{frame}

\begin{frame}
\frametitle{Características}
\framesubtitle{Fenômeno}

\begin{block}

Entidade de interesse do observador que é monitorada e cuja
informação será analisada/filtrada pela rede de sensores. Múltiplos fenômenos
podem ser observados concomitantemente numa rede.

\end{block}

\end{frame}

\begin{frame}
\frametitle{Características}
\framesubtitle{Endereçamento}

\begin{block}

Cada sensor de uma rede de sensores sem fio pode ser acessado individualmente ou não, dependendo da aplicação. 

\end{block} \pause

\begin{exampleblock}

Exemplo: sensores que monitoram funções do corpo humano.

\end{exampleblock}


\end{frame}

\begin{frame}
\frametitle{Características}
\framesubtitle{Agregação de dados}

\begin{block}

Uma rede de sensores com função de agregação de dados pode agregar (ou sumarizar) os dados obtidos por diferentes nós antes do envio à estação base\footnote{Ponto de recepção das mensagens dos sensores da rede.}, reduzindo o tráfego de mensagens.

\end{block}

\end{frame}

\begin{frame}
\frametitle{Características}
\framesubtitle{Mobilidade}

\begin{block}

Em uma rede de sensores sem fio, os sensores devem ter a capacidade de se adaptarem para continuar de acordo com os interesses do observador caso haja mobilidade.
\end{block} \pause

\begin{exampleblock}

Exemplos: nós estáticos utilizados para monitoramento do ritmo cardíaco de um paciente.
Nós móveis utilizados para monitorar condições dentro de um tornado\cite{Twister}.

\end{exampleblock}

\end{frame}

\begin{frame}
\frametitle{Características}
\framesubtitle{Número de sensores}

\begin{block}

Característica que requer atenção especial, pois o número de sensores não deve interferir na eficiência da rede (escalabilidade).
\end{block}

\end{frame}

\begin{frame}
\frametitle{Características}
\framesubtitle{Limitação de consumo de energia}

\begin{block}

Redes de sensores sem fio são extremamente sensíveis ao consumo de energia. Desta forma, o uso de algoritmos, protocolos e aplicações mais robustos e eficientes pode não ser a melhor opção para este tipo de rede. 

\end{block} \pause

\begin{block}

Deve-se levar em consideração o acesso para manutenção dos sensores, que podem estar em áreas remotas. A durabilidade de sua fonte de alimentação determina sua vida útil.

\end{block}

\end{frame}

\begin{frame}
\frametitle{Características}
\framesubtitle{Auto-organização}

\begin{block}

Sensores podem ficar inacessíveis por problemas físicos, como falta de energia, problemas no canal de comunicação sem fio ou ainda por decisão de algum algoritmo de gerenciamento da rede (economia de energia, já existe outro sensor na região que está coletando o dado desejado etc). 

\end{block} \pause

\begin{block}

Pode ocorrer o contrário: sensores inativos passarem à atividade ou a inserção de novos sensores na rede. 
\end{block} \pause

\begin{block}

Em ambos os casos, é necessário que existam ferramentas para a auto-organização da rede, de modo que esta cumpra seus objetivos.

\end{block} \pause

\begin{alertblock}

Essa configuração deve ser feita periodicamente e precisa ser automática, já que o processo manual é totalmente inviável devido a problemas de acesso e escalabilidade. 
\end{alertblock}

\end{frame}

\begin{frame}
\frametitle{Características}
\framesubtitle{Resposta a consultas}

\begin{block}

Uma consulta pode ser solicitada a um nó (sensor) individual ou a um grupo de nós.

\end{block} \pause

\begin{block}

Dependendo do processo de sumarização, pode não ser viável a transmissão dos dados até o nó sorvedouro\footnote{Ponto que gera os interesses iniciais e recebe os dados de interesse} através da rede. 
\end{block} \pause

\begin{block}

Pode ser necessário definir muitos nós sorvedouros que coletarão os dados de uma determinada área, respondendo às consultas referentes aos sensores sob sua responsabilidade.
\end{block} 

\end{frame}

\section{Métricas de desempenho}

\begin{frame}
\frametitle{Métricas de desempenho}
\framesubtitle{Eficiência de energia e vida útil}

\begin{block}

Devido aos sensores possuírem baterias como fonte de energia, é necessário que os protocolos sejam eficientes em relação ao uso de energia, fazendo com que a vida útil do sistema possa ser aumentada. 
\end{block} \pause

\begin{alertblock}

Isso torna a conservação de energia um dos tópicos mais importantes a serem considerados no projeto de uma rede de sensores sem fio.

\end{alertblock}

\end{frame}

\begin{frame}
\frametitle{Métricas de desempenho}
\framesubtitle{Eficiência de energia e vida útil}

\begin{block}

Na comunicação, a maior parte do consumo de energia está na transmissão e na recepção de dados. 

\end{block} \pause

\begin{block}

No processamento, sempre que possível, devem ser empregados métodos de economia de energia nas CPUs dos nós sensores.

\end{block}

\end{frame}

\begin{frame}
\frametitle{Métricas de desempenho}
\framesubtitle{Energias alternativas}

\begin{block}

Pode-se aumentar o tempo de vida útil de um sensor aproveitando algum tipo de energia presente no ambiente, como eólica, solar etc. 

\end{block}

\end{frame}

\begin{frame}
\frametitle{Métricas de desempenho}
\framesubtitle{Latência e precisão}

\begin{block}

Dependendo do fenômeno em observação, pode ser necessário analisá-lo em um certo espaço de tempo (latência), obtendo informações precisas e confiáveis. Para isto o sistema precisa ser eficiente e eficaz.
\end{block} \pause

\begin{block}

Assim, a rede de sensores deve ser estruturada de maneira a obter a precisão e a latência que satisfazem o observador, buscando sempre o uso mínimo de energia. 

\end{block}

\end{frame}

\begin{frame}
\frametitle{Métricas de desempenho}
\framesubtitle{Tolerância a falhas}

\begin{block}

Sensores podem ficar inacessíveis por problemas físicos como a falta de energia, problemas no canal de comunicação sem fio ou por decisão de algum algoritmo de gerenciamento da rede.
\end{block} \pause

\begin{alertblock}

A rede de sensores sem fio deve ser robusta e sobreviver mesmo com a ocorrência de falhas em nós individuais, na rede ou que ocasionem conectividade intermitente. Falhas não catastróficas devem ser transparentes para a aplicação. 
\end{alertblock} \pause

\begin{alertblock}

A falha deve ser tratada como um acontecimento normal, e não como exceção.

\end{alertblock} \pause

\begin{block}

Pode-se alcançar a tolerância a falhas através da replicação de dados, porém esta operação requer energia. Assim, deve-se ter um compromisso entre replicação de dados e eficiência de energia.

\end{block} 

\end{frame}

\begin{frame}
\frametitle{Métricas de desempenho}
\framesubtitle{Escalabilidade}

\begin{block}

Redes de sensores sem fio podem possuir um grande número de nós, o que traz um desafio de escalabilidade. 
\end{block} \pause

\begin{alertblock}

Transmissão de dados redundantes e colisões provocam um gasto de energia desnecessário.  
\end{alertblock} \pause

\begin{alertblock}

O número de nós sensores presentes na rede não deve influenciar o seu desempenho. 
\end{alertblock} \pause

\begin{block}

Escalabilidade exige protocolos de roteamento, endereçamento e agregação de dados escaláveis.

\end{block}

\end{frame}

\begin{frame}
\frametitle{Métricas de desempenho}
\framesubtitle{Exposição dos sensores}

\begin{block}

Medida da capacidade da rede em observar um certo objeto, movendo-se em um caminho arbitrário, em um determinado período de tempo.
 
\end{block} 

\end{frame}

\section{Arquitetura}
\begin{frame}
\frametitle{Arquitetura}

\begin{block}

 \center \textbf{Arquitetura}
 
\end{block}

\end{frame}

\section{Modelos}

\section{Protocolos}

\section{Segurança}

\section{Considerações finais}

\section{Referências}
\begin{frame}
\frametitle{Referências}

\bibliographystyle{ieee}
\bibliography{ref}

\end{frame}

%%%%% Thanks page
\begin{frame}
\frametitle{Agradecimentos}
\vskip20pt

\begin{center}
{\bf \color{alert} Obrigado pela atenção!}
\end{center}

\vskip20pt

\begin{center}
Apresentação disponível em:\\
\url{http:/}
\vskip12pt
\end{center}

\titlepage
\end{frame}

\end{document}