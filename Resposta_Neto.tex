\documentclass[12pt,twoside,a4paper]{article}
\usepackage[brazil]{babel}
\usepackage[utf8]{inputenc}
\usepackage[T1]{fontenc}
\usepackage{timbre-ic}
\usepackage{booktabs}
\usepackage[table]{xcolor}
\usepackage{url}
\usepackage{array}
\usepackage{graphicx}
\usepackage{pifont}


\begin{document}

\vskip 15mm

\begin{center} 
\textbf{Relatório de Projeto  - Tópicos em Redes de Computadores I}

\end{center}

\vskip 5mm

\textbf{Alunos:} Adriano Ricardo Ruggero \textbf{RA:} 144659\\
José Afonso Pinto \textbf{RA:} 860451



\textbf{Professor:} Edmundo R. M. Madeira

\vskip 20mm

\begin{abstract}



\end{abstract}

% resetando configs de layout
\newpage
\pagestyle{plain}
\headheight 0.0cm
\headsep 0.0cm
\footskip 2.2cm

\section{Resposta Seminário}
\label{sec:01}

Em se tratando da topologia \textit{Portland}, a interpretação e resposta a solicitações ARP são delegadas ao chamado ''\textit{fabric manager}''. Segundo os autores, um \textit{switch} de borda intercepta as requisições ARP provenientes dos \textit{hosts} e as encaminha para o ''\textit{fabric manager}''. Este, por sua vez, consulta uma tabela (PMAC - \textit{Pseudo MAC}) que contém pseudo endereços para cada \textit{host} da topologia, além de sua localização na mesma.
Caso o endereço seja localizado na tabela, é retornado o PMAC para o \textit{switch} de borda que originou a solicitação. Este \textit{switch}, por sua vez, cria um \textit{ARP reply} (camada 2) e o retorna ao \textit{host} solicitante.
Se, por qualquer motivo,  o ''\textit{fabric manager}'' não tiver o endereço PMAC correspondente ao solicitado, ele mesmo fará um \textit{broadcast} para todos os \textit{hosts} para obter o mapeamento da topologia (e, assim, gerar o(s) PMAC(s) faltante(s)).
Esta abordagem diminui sensivelmente o tráfego de solicitações ARP em \textit{broadcast} por toda a rede, limitando-as ao ''\textit{fabric manager}'', e apenas ao montar a tabela PMAC (inicialmente ou em caso de falhas)\cite{NiranjanMysore:2009:PSF:1594977.1592575}.


\bibliographystyle{unsrt}
\bibliography{refs_neto}

\end{document}
