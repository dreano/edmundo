\documentclass[12pt,twoside,a4paper]{article}
\usepackage[brazil]{babel}
\usepackage[utf8]{inputenc}
\usepackage[T1]{fontenc}
\usepackage{timbre-ic}
\usepackage{booktabs}
\usepackage[table]{xcolor}
\usepackage{url}
\usepackage{array}
\usepackage{graphicx}
\usepackage{pifont}


\begin{document}

\vskip 15mm

\begin{center} 
\textbf{Seminário  - Topologias de Data Centers - Tópicos em Redes de Computadores I}

\end{center}

\vskip 5mm

\textbf{Aluno:} Adriano Ricardo Ruggero

\textbf{RA:} 144659

\textbf{Professor:} Edmundo R. M. Madeira

\vskip 20mm

\begin{abstract}

O ARP (\textit{Address Resolution Protocol}) é um protocolo usado
para encontrar um endereço de camada 2 (MAC) a partir de
um endereço de camada 3 (IP). O emissor difunde em
\textit{broadcast} um pacote ARP contendo o endereço IP de outro
\textit{host} e espera uma resposta com o endereço MAC respectivo.
Ao conectar milhares de servidores usando uma única rede
de camada 2 devemos evitar o uso de sinalizações em
\textit{broadcast}. Escolha uma das topologias abaixo e explique
como a mesma foi projetada para evitar o \textit{broadcast} do
protocolo ARP:
\begin{itemize}

\item Monsoon (Ref 5 – seção 3.2)

\item VL2 (Ref 4 – seção 4.2)

\item Portland (Ref 6 – seção 3.3)

\end{itemize}

\end{abstract}

% resetando configs de layout
\newpage
\pagestyle{plain}
\headheight 0.0cm
\headsep 0.0cm
\footskip 2.2cm

\section{Resposta Seminário}
\label{sec:01}



\bibliographystyle{unsrt}
\bibliography{refs_neto}

\end{document}
