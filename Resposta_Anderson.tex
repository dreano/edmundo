\documentclass[12pt,twoside,a4paper]{article}
\usepackage[brazil]{babel}
\usepackage[utf8]{inputenc}
\usepackage[T1]{fontenc}
\usepackage{timbre-ic}
\usepackage{booktabs}
\usepackage[table]{xcolor}
\usepackage{url}
\usepackage{array}
\usepackage{graphicx}
\usepackage{pifont}


\begin{document}

\vskip 15mm

\begin{center} 
\textbf{Seminário  - Redes Definidas por Software (SDN),
OpenFlow e outros Controladores de Rede - Tópicos em Redes de Computadores I}

\end{center}

\vskip 5mm

\textbf{Aluno:} Adriano Ricardo Ruggero

\textbf{RA:} 144659

\textbf{Professor:} Edmundo R. M. Madeira

\vskip 20mm

\begin{abstract}

A arquitetura SDN define uma nova forma de estruturar um
sistema em rede, com isso, várias pesquisas são realizadas
para aproveitar essa organização em aplicações de redes de
computadores.

Sabendo disso, pesquise um desses tipos de aplicação que
pode ser melhorado com a utilização de uma SDN, e descreva
sucintamente quais são as vantagens em relação a sua
implementação tradicional.

Para a pesquisa, utilize o artigo de Guedes et al.\cite{SDN}.


\end{abstract}

% resetando configs de layout
\newpage
\pagestyle{plain}
\headheight 0.0cm
\headsep 0.0cm
\footskip 2.2cm

\section{Resposta Seminário}
\label{sec:01}

Uma aplicação que pode se beneficiar em muito com sistemas SDN é a Gerência de Redes.

Na forma tradicional, a gerência de redes não conta com um controle global da rede, paradigma introduzido pelas redes definidas por \textit{software} (SDN - \textit{Software Defined Network}). Este controle global e centralizado permite que novas regras e/ou definições sejam aplicadas com menor esforço por parte do administrador da rede, pois seriam distribuídas pelo controlador da SDN. 

Uma SDN também permite um monitoramento mais claro dos fluxos de interesse (VoIP, por exemplo). 

Alguns projetos de SDN oferecem formas de simplificar o processo de administração de uma rede, incluindo uma base de controle com arquitetura baseada em serviços\cite{OMNI}.

\bibliographystyle{unsrt}
\bibliography{refs}

\end{document}
