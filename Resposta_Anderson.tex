\documentclass[12pt,twoside,a4paper]{article}
\usepackage[brazil]{babel}
\usepackage[utf8]{inputenc}
\usepackage[T1]{fontenc}
\usepackage{timbre-ic}
\usepackage{booktabs}
\usepackage[table]{xcolor}
\usepackage{url}
\usepackage{array}
\usepackage{graphicx}
\usepackage{pifont}


\begin{document}

\vskip 15mm

\begin{center} 
\textbf{Seminário  - Redes Definidas por Software (SDN),
OpenFlow e outros Controladores de Rede - Tópicos em Redes de Computadores I}

\end{center}

\vskip 5mm

\textbf{Aluno:} Adriano Ricardo Ruggero

\textbf{RA:} 144659

\textbf{Professor:} Edmundo R. M. Madeira

\vskip 20mm

\begin{abstract}

A arquitetura SDN define uma nova forma de estruturar um
sistema em rede, com isso, várias pesquisas são realizadas
para aproveitar essa organização em aplicações de redes de
computadores.

Sabendo disso, pesquise um desses tipos de aplicação que
pode ser melhorado com a utilização de uma SDN, e descreva
sucintamente quais são as vantagens em relação a sua
implementação tradicional.

Para a pesquisa, utilize o artigo de Guedes et al.\cite{SDN}.


\end{abstract}

% resetando configs de layout
\newpage
\pagestyle{plain}
\headheight 0.0cm
\headsep 0.0cm
\footskip 2.2cm

\section{Resposta Seminário}
\label{sec:01}



\bibliographystyle{unsrt}
\bibliography{refs}

\end{document}
