\documentclass[12pt,twoside,a4paper]{article}
\usepackage[brazil]{babel}
\usepackage[utf8]{inputenc}
\usepackage[T1]{fontenc}
\usepackage{timbre-ic}
\usepackage{booktabs}
\usepackage[table]{xcolor}
\usepackage{url}
\usepackage{array}
\usepackage{graphicx}
\usepackage{pifont}


\begin{document}

\vskip 15mm

\begin{center} 
\textbf{Seminário  - Redes Veiculares - Tópicos em Redes de Computadores I}

\end{center}

\vskip 5mm

\textbf{Aluno:} Adriano Ricardo Ruggero

\textbf{RA:} 144659

\textbf{Professor:} Edmundo R. M. Madeira

\vskip 20mm

\begin{abstract}

A atribuição dos endereços na Internet atual é feita de forma hierárquica. Em outras palavras, quando um roteador recebe um pacote, ele verifica o endereço de destino e procura na tabela de roteamento o prefixo de maior número de bits iniciais em comum. Esse processo se repete a cada roteador até que seja alcançado o seu destino. Porque essa mesma ideia não é adequada em redes veiculares? Como isso pode ser resolvido?

\end{abstract}

% resetando configs de layout
\newpage
\pagestyle{plain}
\headheight 0.0cm
\headsep 0.0cm
\footskip 2.2cm

\section{Resposta Seminário}
\label{sec:01}

O roteamento utilizado na internet exige que as estações (\textit{hosts}) sejam estáticas, de maneira que a estrutura hierárquica seja mantida. Em uma rede com nós móveis, como ocorre em uma rede veicular, os endereços não são mantidos. Assim, é necessário uma maneira de atribuir endereços automaticamente aos nós, para que a comunicação entre eles possa ocorrer.

Algumas soluções de atribuição de endereços adotadas para redes \textit{ad hoc} móveis já foram consideradas para uso com redes veiculares, entre as quais:

\begin{itemize}

\item Descentralizadas;

\item Melhor esforço;

\item Baseadas em líderes.

\end{itemize}

Contudo, cada uma apresenta problemas inerentes quando utilizadas com redes de alta mobilidade (e. g. redes veiculares).

O protocolo VAC (\textit{Vehicular Address Configuration}), considerando a necessidade de comunicação em tempo real, utiliza a abordagem baseada em líderes, onde qualquer nó da rede possui acesso direto a pelo menos um líder. Os líderes são nós escolhidos aleatoriamente, e fornecem um serviço de DHCP (\textit{Dinamic Host Configuration Protocol)} distribuído, o que assegura a não-duplicidade de endereços IP em um determinado escopo (união das
regiões de alcance de um determinado número de líderes que trocam informações entre
si). A divisão em regiões também colabora para reduzir o tráfego de controle.

Esta abordagem, entretanto, não pode ser utilizada em aplicações que necessitem de endereçamento global, restringindo-se a aplicações para comunicação entre veículos de uma determinada região.

Há, também, a necessidade de se verificar a duplicidade de endereço quando um nó se move entre regiões (escopos) distintas. Assim, a eficiência deste protocolo depende da velocidade relativa entre líderes e os demais nós do escopo, entre outros fatores.

O modo baseado em infraestrutura também pode ser utilizado para atribuir endereços. Neste proposta, é utilizado um servidor DHCP centralizado (que evita problemas como duplicidade de endereços). O ponto negativo desta abordagem é a necessidade de elementos fixos na rede e a utilização de um gerenciador de endereços (governo, concessionárias...).

O GeoSAC (\textit{Geographically Scoped Stateless Address Configuration}) adapta mecanismos inerentes ao protocolo IPv6 para utilização com endereçamento geográfico. Uma camada abaixo da camada de rede é responsável pelo roteamento geográfico, apresentando ao IPv6 uma topologia planificada. Desta forma, o enlace visto pelo IPv6 consistem em nós não diretamente conectados, mas que estão em uma área geográfica relacionada a um ponto de acesso. Este mecanismo pode ser útil para contornar problemas de controle de endereçamento dos nós fora do alcance dos pontos de acesso.

No GeoSAC, o ponto de acesso envia um RA (\textit{Router Advertisement}) para todos os nós de uma determinada área. Em função do mecanismo de encaminhamento geográfico, esta mensagem também alcançará os nós localizados a mais de um salto de distância, mas dentro da área geográfica delimitada.\nocite{Rubinstein}

\bibliographystyle{unsrt}
\bibliography{refs}

\end{document}
