\documentclass[12pt,twoside,a4paper]{article}
\usepackage[brazil]{babel}
\usepackage[utf8]{inputenc}
\usepackage[T1]{fontenc}
\usepackage{timbre-ic}
\usepackage{booktabs}
\usepackage[table]{xcolor}
\usepackage{url}
\usepackage{array}
\usepackage{graphicx}
\usepackage{pifont}


\begin{document}

\vskip 15mm

\begin{center} 
\textbf{Seminário  - LTE e LTE-Advanced - Tópicos em Redes de Computadores I}

\end{center}

\vskip 5mm

\textbf{Aluno:} Adriano Ricardo Ruggero

\textbf{RA:} 144659

\textbf{Professor:} Edmundo R. M. Madeira

\vskip 20mm

\begin{abstract}

Superadas as barreiras da Padronização e da Regulamentação, para implementar uma rede LTE em áreas rurais utilizando a frequência de 450MHz, é preciso ainda considerar alguns desafios. Quais são eles? Escolha um deles e explique-o em poucas linhas.

\end{abstract}

% resetando configs de layout
\newpage
\pagestyle{plain}
\headheight 0.0cm
\headsep 0.0cm
\footskip 2.2cm

\section{Resposta Seminário}
\label{sec:01}


\bibliographystyle{unsrt}
\bibliography{refs_dri}

\end{document}
