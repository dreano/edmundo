\documentclass[12pt,twoside,a4paper]{article}
\usepackage[brazil]{babel}
\usepackage[utf8]{inputenc}
\usepackage[T1]{fontenc}
\usepackage{timbre-ic}
\usepackage{booktabs}
\usepackage[table]{xcolor}
\usepackage{url}
\usepackage{array}
\usepackage{graphicx}
\usepackage{pifont}


\begin{document}

\vskip 15mm

\begin{center} 
\textbf{Seminário  - LTE e LTE-Advanced - Tópicos em Redes de Computadores I}

\end{center}

\vskip 5mm

\textbf{Aluno:} Adriano Ricardo Ruggero

\textbf{RA:} 144659

\textbf{Professor:} Edmundo R. M. Madeira

\vskip 20mm

\begin{abstract}

Superadas as barreiras da Padronização e da Regulamentação, para implementar uma rede LTE em áreas rurais utilizando a frequência de 450MHz é preciso ainda considerar alguns desafios. Quais são eles? Escolha um deles e explique-o em poucas linhas.

\end{abstract}

% resetando configs de layout
\newpage
\pagestyle{plain}
\headheight 0.0cm
\headsep 0.0cm
\footskip 2.2cm

\section{Resposta Seminário}
\label{sec:01}

Um dos desafios a ser considerado quando da implantação de uma rede LTE em zonas rurais é o \textit{backhaul}. Em uma explicação bem rápida, pode-se definir \textit{backhaul} como a porção de uma rede hierárquica de telecomunicações responsável por fazer a ligação entre o núcleo da rede, ou \textit{backbone}, e as subredes periféricas\cite{Backhaul}.

A latência em um circuito de telecomunicações é um fator importante, e pode afetar sensivelmente a experiência do usuário quando da utilização de serviços de banda larga. A latência fim a fim de um circuito é afetada por muitos componentes, como a tecnologia de acesso de rádio. No LTE, isto inclui a rede de acesso (E-UTRAN - \textit{Evolved Universal Mobile Telecommunications System Terrestrial Radio Access Network\cite{EUTRAN}}) e o núcleo da rede (EPC - Entidades do Plano de Controle). Também há que considerar as latências em decorrência do \textit{backhaul} de agregação e da rede de transporte, que conecta os nós da EPC aos nós da E-UTRAN, além das latências entre a EPC e a rede de serviços e a internet.

De maneira a minimizar estes tipos de latência em tais circuitos, torna-se mandatório reduzir as distâncias físicas e o número de nós  intermediários que conectam as E-UTRAN e a EPC\cite{LTE_Rural}.

\bibliographystyle{unsrt}
\bibliography{refs_dri}

\end{document}
