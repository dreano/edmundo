\documentclass[12pt,twoside,a4paper]{article}
\usepackage[brazil]{babel}
\usepackage[utf8]{inputenc}
\usepackage[T1]{fontenc}
\usepackage{timbre-ic}
\usepackage{booktabs}
\usepackage[table]{xcolor}
\usepackage{url}
\usepackage{array}
\usepackage{graphicx}
\usepackage{pifont}


\begin{document}

\vskip 15mm

\begin{center} 
\textbf{Seminário  - Escalonamento em Nuvens Híbridas - Tópicos em Redes de Computadores I}

\end{center}

\vskip 5mm

\textbf{Aluno:} Adriano Ricardo Ruggero

\textbf{RA:} 144659

\textbf{Professor:} Edmundo R. M. Madeira

\vskip 20mm

\begin{abstract}

Todas as informações e estimativas fornecidas como entrada
para o escalonamento são consideradas corretas. Definir tais
informações pode não ser uma tarefa fácil.
Qual o impacto no escalonamento caso essas informações
sejam imprecisas? De forma resumida, cite uma estratégia para
tratar tal problema.


\end{abstract}

% resetando configs de layout
\newpage
\pagestyle{plain}
\headheight 0.0cm
\headsep 0.0cm
\footskip 2.2cm

\section{Resposta Seminário}
\label{sec:01}

Informações e estimativas fornecidas sem a precisão necessária podem levar a perdas na qualidade da informação e gerar incertezas no processo de escalonamento e execução das tarefas. 
Uma estratégia para evitar tal problema pode ser considerar como entrada distribuições estatísticas de tempos e requisitos das aplicações e desempenho dos recursos, e não apenas valores absolutos. Tal critério evitaria abordagens reativas, como realizadas em escalonamento dinâmicos, diminuindo a complexidade do monitoramento do sistema e anulando o impacto de erros nos valores de entrada\cite{Escalonamento}.

\bibliographystyle{unsrt}
\bibliography{refs_erick}

\end{document}
