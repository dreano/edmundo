\documentclass[12pt,twoside,a4paper]{article}
\usepackage[brazil]{babel}
\usepackage[utf8]{inputenc}
\usepackage[T1]{fontenc}
\usepackage{timbre-ic}
\usepackage{booktabs}
\usepackage[table]{xcolor}
\usepackage{url}
\usepackage{array}
\usepackage{graphicx}
\usepackage{pifont}
\usepackage{cancel}


\begin{document}

\vskip 15mm

\begin{center} 
\textbf{Seminário  - Segurança em Nuvens - Tópicos em Redes de Computadores I}

\end{center}

\vskip 5mm

\textbf{Aluno:} Adriano Ricardo Ruggero

\textbf{RA:} 144659

\textbf{Professor:} Edmundo R. M. Madeira

\vskip 20mm

\begin{abstract}

Descreva sucintamente que medidas podem
ser tomadas por um CSP para garantir a
disponibilidade de uma aplicação SaaS.


\end{abstract}

% resetando configs de layout
\newpage
\pagestyle{plain}
\headheight 0.0cm
\headsep 0.0cm
\footskip 2.2cm

\section{Resposta Seminário}
\label{sec:01}

Para aplicações \textit{SaaS}, disponibilidade é essencial. Provedores deste tipo de serviço devem \cancel{tentar} garantir que   seus clientes serão atendidos sempre que solicitarem. Para tal, fazem-se necessárias alterações nos níveis de infra-estrutura e de aplicações para agregar alta disponibilidade e escalabilidade. A adoção de uma arquitetura de multi-camadas também é necessária, apoiada para um grande número de instâncias das aplicações, para um balanceamento de cargas, funcionando em um número considerável de servidores.
A capacidade de tratar falhas de \textit{hardware} e/ou \textit{software} (resiliência), também como ataques DoS (\textit{Denied of Service}), deve ser considerada.
Planos para \textit{business continuity} (continuidade do negócio) e \textit{disaster recovery} (recuperação de desastres) devem ser levadas em conta, caso ocorra uma emergência, assegurando a segurança dos dados das empresas (clientes) e tempo mínimo de inatividade\cite{Subashini:2011:RSS:1889383.1889415}.

\bibliographystyle{unsrt}
\bibliography{refs_zeafonso}

\end{document}
